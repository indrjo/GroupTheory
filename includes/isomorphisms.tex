% !TEX program = lualatex
% !TEX root = ../groups.tex
% !TEX spellcheck = en_GB

\section{Isomorphism Theorems}

Given a group \(G\) and a normal subgroup \(N\) of \(G\), we have the {\em canonical projection}
\[\pi_N : G \to G/N\,, \ \pi_N(x) \coloneq xN .\]

\begin{proposition}[General Isomorphism Theorem]\label{proposition:GrpIso0}
Consider two groups \(G\) and \(H\), a homomorphism \(f : G \to H\) and \(N \subseteq \ker f\) a normal subgroup of \(G\). There exists one and only one homomorphism \(f^\ast : G/N \to H\) such that commutes
\[\begin{tikzcd}[column sep=tiny]
G \ar["f", rr] \ar["{\pi_N}", swap, dr] & & H \\
& G/N \ar["{f^\ast}", swap, ur]
\end{tikzcd}.\]
Furthermore, \(f^\ast\) is surjective if and only if so is \(f\).
\end{proposition}

\begin{proof}
This is the version of Proposition~\ref{proposition:SetIso1} of Group Theory. \(G/N\), with \(N\) normal, partitions \(G\), induced by the relation \(\call_N\) (or \(\calr_N\), which is the same) and, because \(N \subseteq \ker f\), we have that for every \(a, b \in G\) if \(a \call_N b\), then \(f(b) = f(a)\).\newline
You only need to demonstrate \(f^\ast\) is actually a homomorphism, which is immediate: for every \(x, y \in G\)
\[f^\ast((xy)N) = f(xy) = f(x)f(y) = f^\ast(xN)f^\ast(yN) .\qedhere\]
\end{proof}

\begin{proposition}[First Isomorphism Theorem]\label{proposition:GrpIso1}
For \(G\) and \(H\) groups and \(f : G \to H\) homomorphism
\[G/\ker f \cong f(G) .\]
\end{proposition}

\begin{proof}
We use Proposition~\ref{proposition:SetIso1}. A lot of the work is done in the previous proposition. In this case, we have that for every \(a, b \in G\) if \(f(a) = f(b)\) then \(a \call_{\ker f} b\). Hence, by Proposition~\ref{proposition:SetIso1}, we have a (unique) bijection from \(G/\call_{\ker f} = G/\ker f\) to \(f(G)\).
\end{proof}

\begin{proposition}[Classification of cyclic groups]\label{prop:CyclicClassification}
Let \(G\) be a cyclic group. If \(G\) is finite, then \(G \cong \zz/n\zz\) where \(n = \abs G\), otherwise \(G \cong \zz\).
\end{proposition}

\begin{proof}
First of all, \(G = \gen x\) for some \(x \in G\). The function \(f : \zz \to G\), \(f(s) \coloneq x^s\) is a surjective homomorphism, hence \(\zz/\ker f \cong G\). But \(\ker f = n\zz\) for some \(n \in \nn\) by Corollary~\ref{cor:SubgroupsOfZ}. \(\zz/\set{0}\) is infinite since it is isomorphic to \(\zz\), whereas for \(n \in \nn^{\ge 1}\) we have \(\zz/n\zz\) is finite and has \(n\) elements.
\end{proof}

\begin{lemma}\label{lem:CartesianIso}
Let \(G\) be a group and \(H, K\) two subgroups of \(G\) such that:
\begin{enumerate}
\item \(ab = ba\) for every \(a \in H\) and \(b \in K\);
\item \(H \cap K = \set{1}\).
\end{enumerate}
Then \(HK\) is subgroup of \(G\), and \(H \times K \cong HK\).
\end{lemma}

\begin{proof}
We show that \(HK\) is a subgroup of \(G\). Take any pair \(x, y \in HK\): then \(x = h_1k_1\) and \(y = h_2k_2\) for some \(h_1, h_2 \in H\) and \(k_1, k_2 \in K\). So
\begin{align*}
x \inv y & = \underbrace{(h_1k_1) (\inv k_2 \inv h_2) = (h_1k_1) (\inv h_2 \inv k_2)}_{\text{by (1)}} = \\
& = \underbrace{h_1 (k_1 \inv h_2) \inv k_2 = h_1 (\inv h_2 k_1) \inv k_2}_{\text{thanks to (1) again}} = \\
 & = (h_1 \inv h_2) (k_1 \inv k_2) ,
\end{align*}
thus \(x \inv y \in HK\) (by Lemma~\ref{lem:SubgroupsCond}). Now, we prove the function
\[f : H \times K \to HK\,, \ (x, y) \to xy\]
is homomorphism: in fact, for every \((x_1, y_1), (x_2, y_2) \in H \times K\)
\begin{align*}
f((x_1, y_1) (x_2, y_2)) & = f(x_1x_2, y_1y_2) = \\
                         & = \underbrace{(x_1x_2)(y_1y_2) = (x_1y_1)(x_2y_2)}_{\text{by (1)}} = \\
                         & = f(x_1, y_1) f(x_2, y_2) .
\end{align*}
Obviously, \(f\) is surjective. Observe now that for \((a, b) \in H \times K\) if \(ab = 1\), then \(a = \inv b \in K\) and \(b = \inv a \in H\); however, by (2) we must say \(a = b = 1\). We can conclude \(f\) is injective:
\[\ker f = \set{(a, b) \in H \times K \mid ab = 1} = \set{1} .\qedhere\]
\end{proof}

\begin{proposition}[Chinese Remainder Theorem]\label{prop:CRT}
For \(m, n \in \nn^{\ge 2}\) relatively prime numbers and \(G\) abelian group with \(mn\) elements, there exist two subgroups \(H_m\) and \(H_n\) of \(G\) with cardinality \(m\) and \(n\), respectively, such that
\[G \cong H_m \times H_n .\]
\end{proposition}

\begin{proof}
Take the following sets
\[H_m \coloneq \set{x \in G \mid x^m = 1}\,, \ H_n \coloneq \set{x \in G \mid x^n = 1}:\]
since \(G\) is abelian, both are subgroups. Observe both have at least two elements: in fact, by Proposition~\ref{prop:PreCauchysTheorem}, \(H_m\) has some element of order \(p\) for every prime \(p\) dividing \(m\); similarly, \(H_n\) does for the prime divisors of \(n\).\newline
Being \(G\) abelian, one immediately sees the elements of \(H_m\) commutes with the ones of \(H_n\); besides, \(H_m \cap H_n = \set{1}\), since \(m\) and \(n\) are relatively prime. Thus \(H_m \times H_n \cong H_m H_n\) by Lemma~\ref{lem:CartesianIso}. Thanks to Bezout's Lemma, \(am+bn = 1\) for some \(a, b \in \zz\), and consequently
\[x = x^{am+bn} = (x^a)^m (x^b)^n ,\]
where \(x^a \in H_m\) and \(x^b \in H_n\). So \(G = H_m H_n\), and then \(G \cong H_m \times H_n\).\newline
It only remains to examine the size of these subgroups and, to do this, look at the factorization of such cardinalities. If there were a prime number \(p\) that divides either of them, by Proposition~\ref{prop:PreCauchysTheorem} these subgroups would have elements of order \(p\) and then \(H_m \cap H_n\) would not be a singleton. In particular, \(\abs{H_m}\) divides \(m\), because if \(\abs{H_m}\) divided \(n\), then \(H_m\) would be a singleton; similar arguments leads implies \(\abs{H_n}\) divides \(n\). Being \(mn = \abs{H_m} \abs{H_n}\), we can conclude \(H_m\) and \(H_n\) does have \(m\) and \(n\) elements, respectively.
\end{proof}

Probably, you are more familiar with the following version of the Chinese Remainder Theorem, which is a particular consequence of Proposition~\ref{prop:CRT}.

\begin{corollary}
For \(m, n \in \nn^{\ge 2}\) coprime numbers,
\[\zz/mn\zz \cong \zz/m\zz \times \zz/n\zz .\]
\end{corollary}

\begin{proof}
Since \(m\) and \(n\) are relatively prime, by Proposition~\ref{prop:CRT} we have \(\zz/mn\zz \cong H_m \times H_n\) for some subgroups \(H_m\) and \(H_n\) with \(\abs{H_m} = m\) and \(\abs{H_n} = n\). But \(\zz/mn\zz\) is cyclic, hence Proposition~\ref{prop:CyclicOneAndOnlySubgroup} implies there is a unique possibility: \(H_m = \zz/m\zz\) and \(H_n = \zz/n\zz\).
\end{proof}

\begin{exercise}[Important: abelian groups of order \(pq\)]
For \(p\) and \(q\) diverse prime numbers, any abelian group of cardinality \(pq\) is isomorphic to \(\zz/pq\zz\) (in particular, it must be cyclic).
\end{exercise}

\begin{proposition}[Second Isomorphism Theorem]
Let \(G\) be a group. If \(H\) is a subgroup of \(G\) and \(N\) is a normal subgroup of \(G\), then:
\begin{enumerate}
\item \(H \cap N\) is a normal subgroup of \(H\);
\item \(N\) is a subgroup of \(G\) and \(N\) is a normal subgroup of \(HN\);
\item \(H/(H \cap N) \cong HN/N\).
\end{enumerate}
\end{proposition}

\begin{proof}
The proof of (1) and (2) is skipped since it is trivial, so we will prove (3). Take the function
\[f : H \to HN/N\,, \ f(h) \coloneq hN .\]
It is a homomorphism and, since \(N = nN\) for \(n \in N\), is surjective.
Hence, by because of Proposition~\ref{proposition:GrpIso1}, we have \(G/\ker f \cong HN/N\), so we have to calculate the kernel of \(f\):
\[\ker f = \set{g \in H \mid gN = N} = \set{g \in H \mid g \in N} = H \cap N .\qedhere\]
\end{proof}

\begin{proposition}[Third Isomorphism Theorem]
Given a group \(G\) and two normal subgroups \(H\) and \(N\) of \(G\) such that \(N \subseteq H \subseteq G\). Then \(H/N\) is a normal subgroup of \(G/N\) and
\[G/H \cong (G/N)/(H/N) .\]
\end{proposition}

\begin{proof}
The fact that \(H/N\) is a normal subgroup of \(G/N\) is quite immediate. Consider now the homomorphism \(\pi_H\), whose kernel is \(\set{x \in G \mid xH = H} = H\). Since \(N \subseteq H\), by Proposition~\ref{proposition:GrpIso0} there is a homomorphism \(\pi_H^\ast : G/N \to G/H\) such that
\[\begin{tikzcd}[column sep=tiny]
G \ar["{\pi_H}", rr] \ar["\pi_N", swap, dr] & & G/H \\
& G/N \ar["{\pi_H^\ast}", swap, ur]
\end{tikzcd}\]
commutes. Because \(\pi_H\) is surjective \(\pi_H^\ast\) is surjective too, and then by Proposition~\ref{proposition:GrpIso1} we have \((G/N)/\ker \pi_H^\ast \cong G/H\), where
\begin{align*}
\ker \pi_H^\ast &= \set{xN \in G/N \mid \pi_H^\ast(xN) = H} = \\
                &= \set{xN \in G/N \mid xH = H} = \\
                &= \set{xN \mid x \in H} = H/N .\qedhere
\end{align*}
\end{proof}

\begin{exercise}
For \(G\) group and \(N\) normal subgroup of \(G\) such that \(G/N\) is an infinite cyclic group show that for every \(n \in \nn^{\ge 1}\) there exists a normal subgroup \(H\) of \(G\) such that \([G:H] = n\).
\end{exercise}

\begin{exercise}
Let \(G\) be a group and \(H, K\) two of its finite subgroups with the following properties: \(ab = ba\) for every \(a \in H\) and \(b \in K\).
Show that
\[\frac{\abs H \abs K}{\abs{H \cap K}} = \abs{HK} .\]
\end{exercise}
