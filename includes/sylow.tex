% !TEX program = lualatex
% !TEX spellcheck = en_GB
% !TEX root = ../groups.tex

\section{Sylow Theorem}

\begin{lemma}\label{lem:PreSylowTheoremI}
Let \(G\) be a finite group. For every prime number \(p\) and \(r \in \nn^{\ge 1}\) such that \(p^r\) divides \(\abs G\) there exists a subgroup of \(G\) of cardinality \(p^r\).
\end{lemma}

\begin{proof}[Proof, with \(G\) abelian] We use induction on the cardinality of \(G\). If \(G\) has \(2\) elements, the statement is true. Thanks to Proposition~\ref{prop:PreCauchysTheorem}, there exists a cyclic subgroup \(H\) of \(G\) with order \(p\). Since \(G\) is abelian, \(H\) is abelian (thus it is normal too), and so we have the abelian group \(G/H\) that has cardinality multiple of \(p^{r-1}\) (Proposition~\ref{prop:LagrangesTheorem}) and less then \(\abs G\). By inductive hypothesis, we there is a subgroup \(K\) of \(G/H\) that has \(p^{r-1}\) elements; besides, \(K = K'/H\) for some \(K'\) subgroup of \(G\). We can conclude \(\abs {K'} = p^{r}\), again by Proposition~\ref{prop:LagrangesTheorem}.
\end{proof}

\begin{proof}[Proof of the general case]
Again by induction on \(\abs G\). The case in which \(G\) has \(2\) elements is trivial. If \(G\) is abelian, we fall back into the previous situation. Then, let \(\calz G\) be a proper subgroup of \(G\) and assume \(\abs G = p^r k\) for some \(k \in \nn\) . Let \(R \subseteq G\) such that \(\set{[x]_G \mid x \in R}\) is a partition of \(G\): by Proposition~\ref{prop:ClassFormula} we have
\[p^r k = \abs{\mathcal Z G} + \sum_{x \in R \setminus \mathcal Z G} [G:C_G(x)] ,\]
where for \(x \in R \setminus \calz G\) we have \(\abs {C_G (x)} = p^{r_x} h_x\) for some \(r_x, h_x \in \nn^{\ge 1}\) such that it divides \(p^r k\); without loss of generality, we can suppose \(p\) does not divide \(h_x\). If there is an \(a \in R \setminus \calz G\) such that \(p^r\) does not divide \([G:C_G(a)]\), we have actually \(\abs{C_G(a)} = p^r h_a\), which is less than \(p^r k\) since \(G\) is not abelian. By induction, \(C_G(a)\) has a subgroup with \(p^r\) elements. Otherwise, if every \(a \in R \setminus \calz G\) is such that \(p^r\) divides \([G:C_G(a)]\), then \(p\) does divide \(\calz G\). Now, thanks to Proposition~\ref{prop:PreCauchysTheorem}, there exists a cyclic subgroup \(H\) of \(\calz G\) with order \(p\). We have then the quotient \(G/H\), since \(H\) is normal; it has cardinality multiple of \(p^{r-1}\). By induction, there exists a subgroup \(K/H\) of \(G/H\) with \(p^{r-1}\) elements, so we can conclude \(\abs K = p^r\).
\end{proof}

From Lemma~\ref{lem:PreSylowTheoremI} comes the generalization of Proposition~\ref{prop:PreCauchysTheorem}, that is Lemma~\ref{lem:PreSylowTheoremI} with \(r=1\).

\begin{proposition}[Cauchy's Theorem]\label{prop:CauchysTheorem}
Let \(G\) be a finite group. Then for every prime \(p \in \nn\) that divides \(\abs G\) there exists \(x \in G\) such that \(\ord x = p\).
\end{proposition}

\begin{lemma}\label{lem:PreSylowTheoremII}
Let \(H\) be a group of order \(p^r\), for some prime \(p\) and \(r \in \nn^{\ge 1}\), and \(\phi\) an action of \(H\) on a set \(X\); consider \(X_0 \coloneq \set{x \in X \mid \stab_\phi x = G}\). Then
\[\abs X \equiv \abs{X_0} \bmod p .\]
\end{lemma}

\begin{proof}
\(X\) is partitioned by the orbits of its elements. In particular, by Proposition~\ref{prop:OrbsStabsCosets}, the non banal orbits are powers of \(p\).
\end{proof}

\begin{proposition}[Sylow Theorem]\label{prop:SylowTheorem}
Let \(p\) be a prime number and  \(G\) a group with \(\abs G = p^rk\), for \(r, k \in \nn^{\ge 1}\) such that \(p\) does not divide \(k\).
\begin{enumerate}
\item There exists a subgroup of \(G\) with \(p^r\) elements.
\item Let \(S\) and \(H\) be subgroups of \(G\) with cardinality \(p^r\) and \(p^n\) respectively. Then \(\inv g H g \subseteq P\) for some \(g \in G\). In particular, two any subgroups of \(G\) with \(p^r\) elements are conjugated.
\item Let \(s_p\) be the number of subgroups of \(G\) with \(p^r\) elements. Then %\(s_p \equiv 1 \bmod p\) and \(s_p\) divides \(k\).
\[\begin{cases}
s_p \equiv 1 \bmod p \\
s_p \text{ divides } k .
\end{cases}\]
\end{enumerate}
\end{proposition}

\begin{proof}
\begin{enumerate}
\item Immediate consequence of Lemma~\ref{lem:PreSylowTheoremI}.
%
\item Consider the action
\[\phi : H \to \cals(G/\call_S)\,,\ \phi_h (C) \coloneq hC .\]
By Lemma~\ref{lem:PreSylowTheoremII}, we have
\[[G:S] \equiv \abs\Omega \bmod p ,\]
where
\[\Omega \coloneq \set{gS \in G/\call_S \mid \stab_\phi (gS) = H} .\]
By assumption, \(p\) does not divide \([G:S]\), hence it neither divides \(\abs\Omega\). In particular, \(\abs\Omega \ne 0\), so there exists \(gS \in G/\call_S\) such that \(\phi_h(gS) = hgS = gS\) for every \(h \in H\). That is, \((\inv g h g) S = S\) for every \(h \in H\), and then \(\inv g H g \subseteq S\).
%
\item Let \(X\) be the family of the subgroups of \(G\) with cardinality \(p^r\), and consider the action of \(G\) on \(X\)
\[\eta : G \to \cals X\,, \ \eta_g (S) = \inv g S g .\]
By the first part of this theorem, there exists \(S \in X\) such that \(\orb_\eta (S) = X\). Hence, using Proposition~\ref{prop:OrbsStabsCosets},
\[s_p = \abs{\orb_\eta S} = \frac{\abs G}{\abs{\stab_\eta S}}.\]
But \(\abs{\stab_\eta S} \ge \abs S = p^r\) and \(\abs{\stab_\eta S}\) divides \(\abs G = p^r k\), hence (it is crucial that \(p\) is prime) \(s_p\) does divide \(k\). Besides,
\[X_0 \coloneq \set{H \in X \mid \inv g H g = H \text{ for every } g \in G}\]
is a singleton: it has at least one element because
\[s_p = \abs X \equiv \abs{X_0} \bmod p ,\]
so if it were empty, \(s_p\) would be a multiple of \(p\), absurd; \(X_0\) has at most one element, since two any \(H_1, H_2 \in X_0\) are conjugated and then, by how \(X_0\) is defined, equal. Thanks to Lemma~\ref{lem:PreSylowTheoremII}, we conclude \(s_p \equiv 1 \bmod p\).\qedhere
\end{enumerate}
\end{proof}
